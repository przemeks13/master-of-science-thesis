\chapter{Zestawienie pomiarów} % Main appendix title

\label{app:suplementA}

\begin{table}[htpb]
    \caption{Parametry sygnałów otrzymanych dla różnych
     przetworników PVDF w układzie konstrukcyjnym Acc oraz Acf}
    \label{fig:results_sensor_selection}

    \centering
    \pgfplotstabletypeset[
    col sep = comma,
    columns = {Np, probka, vpp, dt, jakość, Uwagi },
    sort,
    sort cmp={string <},
    sort key=probka,
    every head row/.style={before row=\toprule,after row=\midrule},
    every last row/.style={after row=\bottomrule},
    % display columns/0/.style={string type,column name={}}
    display columns/1/.style={string type,column name={}},
    % display columns/2/.style={string type,column name={}}
    % display columns/3/.style={string type,column name={}}
    % display columns/4/.style={string type,column name={}}
    display columns/5/.style={string type,column name={}}
    ]
    {selekcja_czujnika.csv}
\end{table}

\begin{table}[htpb]
    \caption{Wartości sygnałów otrzymanych z przetwornika 1.1.
     dla różnych układów konstrukcyjnych}
    \label{fig:results_11_geometry}

    \centering
    \pgfplotstabletypeset[
    col sep = comma,
    columns = {Np, probka, vpp, dt, jakość, Uwagi },
    sort,
    sort cmp={int <},
    sort key=Np,
    every head row/.style={before row=\toprule,after row=\midrule},
    every last row/.style={after row=\bottomrule},
    % display columns/0/.style={string type,column name={}}
    display columns/1/.style={string type,column name={}},
    % display columns/2/.style={string type,column name={}}
    % display columns/3/.style={string type,column name={}}
    % display columns/4/.style={string type,column name={}}
    display columns/5/.style={string type,column name={}}
    ]
    {geometry_optymization_11.csv}
\end{table}

\begin{table}[htpb]
    \caption{Wartości sygnałów otrzymanych z przetwornika 6.
    dla różnych układów konstrukcyjnych}
    \label{fig:results_6_geomrtry}

    \centering
    \pgfplotstabletypeset[
    col sep = comma,
    columns = {Np, probka, vpp, dt, jakość, Uwagi },
    sort,
    sort cmp={int <},
    sort key=Np,
    every head row/.style={before row=\toprule,after row=\midrule},
    every last row/.style={after row=\bottomrule},
    % display columns/0/.style={string type,column name={}}
    display columns/1/.style={string type,column name={}},
    % display columns/2/.style={string type,column name={}}
    % display columns/3/.style={string type,column name={}}
    % display columns/4/.style={string type,column name={}}
    display columns/5/.style={string type,column name={}}
    ]
    {geometry_optymalization_6.csv}
\end{table}

\begin{table}[htpb]
    \caption{Wartości sygnałów otrzymanych z przetwornika 1.1.
    w układzie konstrukcyjnym rBcc dla różnych wartości energii wymuszenia}
    \label{fig:results_11rBcc_char}

    \centering
    \pgfplotstabletypeset[
    col sep = comma,
    columns = {Np, x, Ec, vpp, dt, jakość, Uwagi },
    sort,
    sort cmp={int <},
    sort key=Np,
    every head row/.style={before row=\toprule,after row=\midrule},
    every last row/.style={after row=\bottomrule},
    % display columns/0/.style={string type,column name={}}
    % display columns/1/.style={string type,column name={}},
    % display columns/2/.style={string type,column name={}}
    % display columns/3/.style={string type,column name={}}
    % display columns/4/.style={string type,column name={}}
    display columns/6/.style={string type,column name={}}
    ]
    {1.1rBcc_chart.csv}
\end{table}

\begin{table}[htpb]
    \caption{Wartości sygnałów otrzymanych z przetwornika 1.1.
    w układzie konstrukcyjnym S dla różnych wartości energii wymuszenia}
    \label{fig:results_11S_char}

    \centering
    \pgfplotstabletypeset[
    col sep = comma,
    columns = {Np, x, Ec, vpp, dt, jakość, Uwagi },
    sort,
    sort cmp={int <},
    sort key=Np,
    every head row/.style={before row=\toprule,after row=\midrule},
    every last row/.style={after row=\bottomrule},
    % display columns/0/.style={string type,column name={}}
    % display columns/1/.style={string type,column name={}},
    % display columns/2/.style={string type,column name={}}
    % display columns/3/.style={string type,column name={}}
    % display columns/4/.style={string type,column name={}}
    display columns/6/.style={string type,column name={}}
    ]
    {1.1S_chart.csv}
\end{table}

\begin{table}[htpb]
    \caption{Wartości sygnałów otrzymanych z przetwornika 6.
    w układzie konstrukcyjnym rBcc dla różnych wartości energii wymuszenia}
    \label{fig:results_6rBcc_center_char}

    \centering
    \pgfplotstabletypeset[
    col sep = comma,
    columns = {Np, x, Ec, vpp, dt, jakość, Uwagi },
    sort,
    sort cmp={int <},
    sort key=Np,
    every head row/.style={before row=\toprule,after row=\midrule},
    every last row/.style={after row=\bottomrule},
    % display columns/0/.style={string type,column name={}}
    % display columns/1/.style={string type,column name={}},
    % display columns/2/.style={string type,column name={}}
    % display columns/3/.style={string type,column name={}}
    % display columns/4/.style={string type,column name={}}
    display columns/6/.style={string type,column name={}}
    ]
    {6rB_center_chart.csv}
\end{table}