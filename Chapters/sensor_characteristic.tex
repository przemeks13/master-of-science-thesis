\chapter{Charakterystyki wyjściowe przetworników}
\label{sec:sensor_characteristic}

\section{Założenia}
\label{sec:sensor_characteristic_assumptions}

W rozdziale \ref{sec:construction_optymization} opisano metodę oraz przebieg optymalizacji 
konstrukcji projektowanego przetwornika piezoelektrycznego. Konkluzją powyższego jest wybór
konkretnego, najbardziej optymalnego układu konstrukcyjnego oraz wyznaczenie charakterystyk
mechaniczno-elektrycznych. Oznaczna to zaprezentowanie wielkości elektrycznych 
sparametryzowanych wielkością mechaniczną. W kolejnych podrozdziałach zostaną przedstawione
oraz krótko skomentowane zależności $V_{pp}(E_k)$ oraz $dt(E_k)$ dla układów konstrukcyjnych
\textbf{S z czujnikiem 1.1., rBcc z czujnikiem 1.1. oraz czujnikiem 6}. Zdecydowano się 
zbadać 3 różne konstrukcje, ponieważ krok optymalizacji wskazał wyłącznie niewyraźnego
lidera. Ponadto, co wykazano w rozdziale \ref{sec:construction_optymization}, więcej niż
jedno z przewidzianych rozwiązań konstrukcyjnych nadaje się do zastosowania w założonym
układzie (patrz: rozdział \ref{sec:assumptions}). W wyborze kierowano się również
oceną możliwości montażu tak przygotowanego przetwornika.

\section{Otrzymane charakterkystyki}
\label{sec:sensor_characteristic_td}

\indent Poniżej na rys. \ref{fig:tdEc_char} oraz \ref{fig:vppEc_char} przedstawiono
charakterystyki wyjściowe dla wybranych konstrukcji przetworników piezoelektrycznych.
Uzasadnienie wyboru pokazano w sekcji \ref{sec:sensor_characteristic_assumptions}. 
Otrzymane charakterystyki zostaną krótko skomentowane. Charakterystyka 
$t_d(E_C)$ (patrz: rys. \ref{fig:tdEc_char}) wykazuje stałość czasu tłumienia
od energii kinetycznej ciała M tuż przed zderzeniem. Jedynie dla układu $1.1rBcc$
początkowe wartości $t_d$ są  nieco wyższe od pozostałych. Energia zderzenia, a zatem
napięcie $u_{pp}$ były wtedy na tyle małe, że czas trwania sygnału przy pomocy
użytych przyrządów pomiarowych (kursorów na oscyloskopie) był trudny do odczytania.
Można zatem stwoierdzić, że te dwie próbki są grubym błędem pomiarowym i prawdopodobnie
zadziałał tu efekt autosugestii przy wykonywaniu pomiarów. Głównym wnioskiem z 
otrzymanych zależności $t_d(E_C)$ jest to, że czas tłumienia sygnału nie jest zależny
od siły uderzenia w przetwornik, a od stałej tłumienia konstrukcji przetwornika. 

\indent Odnosząc się do zależności $u_{pp}(E_C)$ ( patrz: \ref{fig:vppEc_char})
należy zauważyć, że wraz ze wzrostem energii uderzenia rośnie wartość napięcia
międzyszczytowego $u_{pp}$. Interesujący jest kształt tendencji rosnącej.
Niepodważalnym jest, że przy wyższych wartościach $E_C$ na okreśclony przyrost
$\delta E_C$ przypada mniejszy przyrost $\delta u_{pp}$. Mówiąc krócej, pochodna
$\frac{ du_{pp}}{dE_C}$ maleje wraz ze wzrostem $E_C$. Wniosek jaki należy wyciągnąć
z tych zależności to odnalezienie przyczyny takiego kształtu. Otóż poruszono to już w
podrozdziale \ref{sec:signal_description}. Energia kinetyczna ciała M przed 
zderzeniem jesst wprost proporcjonalna do kwadratu napięcia międzyszczytowego
(patrz: zależność \ref{eq:energy_prop}). Te natomiast jest wprost proporcjonalne do
maksymalnego wychylenia, czyli amplitudy ugięcia belki.

\indent Jeszcze krótko na temat zakresów uzyskanych wartości. Czas tłumienia sygnału
we wszystkich 3 przypadkach w całym przedziale $E_C$ mieści się w wymaganiach postawionych
w rozdziale \ref{sec:assumptions} dotyczących częstotliwości występujacych w układzie
rzeczywistym wymuszeń mechanicznych. Przy czym najlepiej w tym aspekcie prezentuje się
układ 1.1rBcc, czyli model rury z przyklejonym od zewnątrz przetwornikiem PVDF 1.1.
Odnośnie napięcia $u_{pp}$ uzyskano przedział wartości $0 \div 22 V$. Głównym problemem
nie będą tu wysokie, lecz niskie wartości napięcia, przy małych energiach zderzeń. 
Na stanowisku laboratoryjnym wartości kilkuset $mV$ nie są problematyczne do detekcji.
W układzie rzyczywistym, gdzie występują szumy może stać się to uciążliwe i należy mieć
to na uwadze.

\pgfplotsset{width=\linewidth,compat=1.3}

\begin{figure}[htbp]
\centering
    \begin{tikzpicture}
%    \pgfplotsset{
%    scale only axis,
%    scaled x ticks=base 10:3,
%    xmin=0, xmax=0.06
%}
      \begin{axis}[
          width=\linewidth, % Scale the plot to \linewidth
          grid=major, % Display a grid
          grid style={dashed,gray!30}, % Set the style
          xlabel=$E_C$ , % Set the labels
          ylabel=$t_d$,
          x unit=\si{\milli\joule}, % Set the respective units
          y unit=\si{\milli\second},
          ymin = 0,
          xmin = 0,
          mark size = 4pt,
          title =Charakterystyka $t_d(E_C)$
          ]
          \addplot[smooth, only marks, mark=*,red] table[x=Ec,y=dt,col sep=comma] {1.1rBcc_chart.csv};
          \addplot[smooth, only marks, mark=x,green] table[x=Ec,y=dt,col sep=comma] {1.1S_chart.csv};
          \addplot[smooth, only marks, mark=o, blue] table[x=Ec,y=dt,col sep=comma] {6rB_center_chart.csv};
          \legend{{$1.1rBcc$}, {$1.1S$}, {$6rBcc\_center$}}
        \end{axis}
    \end{tikzpicture}
\caption{Charakterystyka czasu trwania sygnału od energii ciała M przed zderzeniem.}
\label{fig:tdEc_char}
\end{figure}


\begin{figure}[htbp]
    \centering
        \begin{tikzpicture}
    %    \pgfplotsset{
    %    scale only axis,
    %    scaled x ticks=base 10:3,
    %    xmin=0, xmax=0.06
    %}
          \begin{axis}[
              width=\linewidth, % Scale the plot to \linewidth
              grid=major, % Display a grid
              grid style={dashed,gray!30}, % Set the style
              xlabel=$E_C$ , % Set the labels
              ylabel=$t_d$,
              x unit=\si{\milli\joule}, % Set the respective units
              y unit=\si{\milli\second},
              ymin = 0,
              xmin = 0,
              mark size = 4pt,
              title =Charakterystyka $u_{pp}(E_C)$,
              legend style = {at={(0.03,0.97)}, anchor = north west}
              ]
              \addplot[smooth, only marks, mark=*,red] table[x=Ec,y=vpp,col sep=comma] {1.1rBcc_chart.csv};
              \addplot[smooth, only marks, mark=x,green] table[x=Ec,y=vpp,col sep=comma] {1.1S_chart.csv};
              \addplot[smooth, only marks, mark=o, blue] table[x=Ec,y=vpp,col sep=comma] {6rB_center_chart.csv};
              \legend{{$1.1rBcc$}, {$1.1S$}, {$6rBcc\_center$}}
            \end{axis}
        \end{tikzpicture}
    \caption{Charakterystyka początkowego napięcia międzyszczytowego sygnału od 
    energii ciała M przed zderzeniem.}
    \label{fig:vppEc_char}
    \end{figure}
