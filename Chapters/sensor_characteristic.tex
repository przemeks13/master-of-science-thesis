\chapter{Charakterystyki przetworników}
\label{sec:sensor_characteristic}

\section{Założenia}
\label{sec:sensor_characteristic_assumptions}

W rozdziale \ref{sec:construction_optymization} opisano metodę oraz przebieg optymalizacji 
konstrukcji projektowanego przetwornika piezoelektrycznego. Konkluzją powyższego jest wybór
konkretnego, najbardziej optymalnego układu konstrukcyjnego oraz wyznaczenie charakterystyk
mechaniczno-elektrycznych. Oznaczna to zaprezentowanie wielkości elektrycznych 
sparametryzowanych wielkością mechaniczną. W kolejnych podrozdziałach zostaną przedstawione
oraz krótko skomentowane zależności $V_{pp}(E_k)$ oraz $dt(E_k)$ dla układów konstrukcyjnych
\textbf{S z czujnikiem 1.1., rBcc z czujnikiem 1.1. oraz czujnikiem 6}. Zdecydowano się 
zbadać 3 różne konstrukcje, ponieważ krok optymalizacji wskazał wyłącznie niewyraźnego
lidera. Ponadto, co wykazano w rozdziale \ref{sec:construction_optymization}, więcej niż
jedno z przewidzianych rozwiązań konstrukcyjnych nadaje się do zastosowania w założonym
układzie (patrz: rozdział \ref{sec:assumptions}). W wyborze kierowano się również
oceną możliwości montażu tak przygotowanego przetwornika.

\section{Badanie konsrukcji S z czujnikiem 1.1.}
\label{sec:sensor_characteristic_11S}

\section{Badanie konstrukcji rBcc z czujnikiem 1.1.}
\label{sec:sensor_characteristic_11rBcc}

\section{Badanie konstrukcji rBcc z czujnikiem 6}
\label{sec:sensor_characteristic_6rBcc}
