%sekcja wyboru czujnika piezo
\chapter{Selekcja czujnika}
\label{sec:sensor_selection}

\section{Przegląd rynku piezoelektryków}
\label{sec:piezoelectric_research}

Równolegle z pracami nad stanowiskiem badawczym przeprowadzono przegląd dostępnych
na rynku przetworników piezoelektrycznych. Uwagę skoncentrowano przede wszystkim 
na przetwornikach PVDF
\footnote{Polifluorek winylidenu - 
materiał piezoelektryczny charakteryzujący się elastycznością.}

Do badań wybrano 6 konkretnych produktów (patrz: Rys.\ref{fig:sensors}.):

\begin{enumerate}
\item Czujnik TODO
\item Czujnik TODO
\item Czujnik TODO
\item Czujnik TODO
\item Czujnik TODO
\item Czujnik TODO
\end{enumerate}


\begin{figure}[tbhp]
\centering
\includegraphics[width=\linewidth]{pictures/sensors.jpg}
\caption{Badane przetworniki piezoelektryczne.}
\label{fig:sensors}
\end{figure}

\pagebreak

\section{Wybór optymalnego przetwornika}
\label{sec:optimal_piezoelectric_selection}

Aby dokonać selekcji przetwornika wybrano najprostszy, belkowy układ konstrukcyjny 
(patrz: Rys. \ref{fig:sensor_sel_geometry}). Został on wykonany z fragmentu 
tworzywa sztucznego. Układ oznaczono tak, aby można było ustawić go ponownie 
w tej samej pozycji po wymianie sensora. Sensor przyklejano na taśmę dwustronną 
o grubości ok. $ d = 1.5 mm$ w ustalonym uprzednio miejscu. Na temat
belkowych przetworników piezoelektrycznych można dowiedzieć się więcej 
z pozycji literaturowej \cite{belkowy_sensor}. 

\begin{figure}[tbhp]
\centering
\fbox{TUTAJ RYSUNEK GEOMETRII CZUJNIKA }%\includegraphics[width=\linewidth]{sample}}
\caption{Konstrukcja pozwalająca na selekcję przetwornika piezoelektrycznego.}
\label{fig:sensor_sel_geometry}
\end{figure}

\pagebreak 
W celu uzyskania szerszego spektrum danych dla każdego czujniaka wykonano
poamiary w dwóch wariantach: 
\begin{enumerate}
\item stała podpora na jednym z końców układu przetwornika, drugi koniec swobodny,\nopagebreak
\item stała podpora na jednym z końców układu przetwornika, drugi koniec z 
amortyzatorem z gąbki.
\end{enumerate}

Dla każdego pomiaru przeprowadzono po 10 prób, co pozwoliło zebrać dane zawarte w tabeli
\ref{fig:results_sensor_selection}. Dane zostały wstępnie obrobione, to znaczy zestawiono
je w ujęciu statystycznym.
Parametry, jakie wzięto pod uwagę przy ocenie jakości sygnału, to czas trwania 
( od inicjalizacji do wygaszenia ) oraz wartość napięcia międzyszczytowego oznaczanych 
dalej odpowiednio $t_d$ i $V_{pp}$. Ich porównanie po statystycznej analizie 
przedstawiono na Rys.\ref{fig:sensor_statistic}. Wykres oceny jakości sygnału w skali $0\div5$ 
oznacza subiektywną ocenę poziomu zniekształcenia otrzymanego sygnału. Ocena 0 oznacza mocno
zniekształcony sygnał (patrz: rys. \ref{fig:distorted_signal}) , natomiast 5 sygnał nieodkształcony 
lub odkszktałcony w minimalnym stopniu (patrz: rys. \ref{fig:ideal_signal}). 
Należy nadmienić, że podczas badań zwracano uwagę na sposób przyklejenia przetwornika do plastikowego 
płaskownika tak, aby nie fałszowało to przeprowadzanych pomiarów. 
%TODO Tu przydałby się rozkład widmowy sygnału


\pgfplotstableread[col sep=comma]{selekcja_czujnika.csv}\sensorSelTab


\begin{figure}[htbp]
  \centering
  
\pgfplotsset{compat=1.12}
\begin{tikzpicture}

  \begin{groupplot}[
    group style={
        group name=my fancy plots,
        group size=1 by 3,
        xticklabels at=edge bottom,
        xlabels at = edge bottom,
        vertical sep=25pt
    },
    ybar,
    /pgf/bar width=1,
    ymin=0,
    width=\textwidth,
    height=0.4\textwidth,
    axis x line=bottom,
    enlarge x limits=0.25,
    major x tick style = transparent,
    xtick=data,
    xticklabels from table={\sensorSelTab}{probka},
    xticklabel style={rotate=60, font=\scriptsize},
    yticklabel style={font=\scriptsize},
    xlabel style={at={(-0.1,-0.05)}, font=\scriptsize},
    xlabel={próbka},
    ylabel style={font=\scriptsize},
    title style={at={(0.5,1.0)}, font=\footnotesize}, 
    grid = major,
    table/x expr=\coordindex,
    table/col sep=comma,
]

\nextgroupplot[ylabel={$V_{pp}[V]$},
               ymax=30,
               ytick distance = 10,
               title={Maksymalne napięcie międzyszczytowe},
              ]
\addplot [fill = red] table[y=vpp]{selekcja_czujnika.csv};     

\nextgroupplot[ymode=log,
               ylabel={$t_d[ms]$},
               title={Czas trwania sygnału},
               ]
\addplot [fill = green] table[y=dt]{selekcja_czujnika.csv};;
\nextgroupplot[ymax=5,
               ylabel={$Q[b.j.]$},
               ytick distance = 1,
               title={Ocena jakości sygnału},
               ]
\addplot [fill = blue] table[y=jakość]{selekcja_czujnika.csv};                 
\end{groupplot}

\end{tikzpicture}

\caption{Zestawienie parametrów sygnałów dla poszczególnych czujników na
 podstawie tablicy \ref{fig:results_sensor_selection}.}
\label{fig:sensor_statistic}
\end{figure}


\indent Na podstawie rys.\ref{fig:sensor_statistic} można stwierdzić, że najwyższą wartość
 uzyskanego napięcia $V_{pp}$ otrzymano dla czujnika oznaczonego numerem \textbf{6}.
 Nie gorzej wypadł w tej klasyfikacji przetwornik \textbf{1.1.} Biorąc pod uwagę
 niepewność pomiarową można stwierdzić, że prowadzenie czujnika 6. jest pomijalnie małe.
 Ranking 5 próbek o najlepszych rezultatach przedstawiono dla czytelności w tablicy 
 \ref{fig:sensor_selection_rank_vpp}. 
 \indent Szacuje się, że wyższa wartość $V_{pp}$ będzie bradziej odporna na szumy pochodzące 
 z maszyny, na której sensor byłby umieszczony. Zgodnie z założeniami przeznaczeniem 
 sensora jest przecież licznik impulsów mechanicznych. Z dużym prawdopodobieństwem 
 można uznać, że wyzwalanie licznika następować będzie na jednym ze zboczy pierwszej 
 półfali sygnału (patrz przebieg na rys. \ref{fig:scope_with_silencer}). Nieco inaczej będzie to wyglądać w przypadku sygnału wyprostowanego
 lub nawet stabilizowanego. Takie przekształcenie ułatwi identyfikację wystąpienia 
 wymuszenia. 

 \indent Spoglądając ponownie na wykres $V_{pp}$ na rys. \ref{fig:sensor_statistic}
 można dostrzec pewną ciekawą prawidłowość. Otóż, pomijając czujnik 6., którego sposób
 montażu był nieco inny, wartości napięcia międzyszczytowego dla układów z jedną podporą
 tłumiącą były niższe niż ich odpowiedniki w układach bez drugiej podpory. Wyjaśnieniem
 tego zjawiska jest fakt, iż podpora tłumiąca przecidziała wymuszeniu mechanicznemu i 
 zmniejsza wychylenie belki przetwornika w stosunku do belki wyłącznie z jedną podporą.
 Uzupełnienie informacji w tym temacie zawiera pozycja \cite{belkowy_sensor}.


\begin{table}[h]
  \caption{Ranking optymalizacji pod względem napięcia międzyszczytowego $V_{pp}$}
  \label{fig:sensor_selection_rank_vpp}

  \centering
  \pgfplotstabletypeset[
  col sep = comma,
  columns = {Np,probka,vpp},
  every head row/.style={before row=\toprule,after row=\midrule},
  every last row/.style={after row=\bottomrule},
  display columns/0/.style={column name={Np.}, column type = {r}},
  display columns/1/.style={string type,column name={Próbka}, column type = {l}},
  display columns/2/.style={column name={$V_{pp} [\si{\volt}]$},
  precision=2,fixed zerofill, column type = {r}, fixed},
  skip rows between index={4}{20}
  ]
  {selekcja_czujnika_vpp.csv}
\end{table}
 
\indent Odnosząc się do czasu trwania sygnału jednoznacznie można stwierdzić, 
że im krótszy tym lepszy. Pozwala to uzyskać wyższą częstotliwość wymuszeń mechanicznych 
bez wystąpienia zjawiska aliasingu. Oczywiście detekcja zbyt krótkiego sygnału może być 
problematyczna. Krótkotrwały sygnał wymaga dużej częstotliwości przełączania układu bramkującego.
Przekłąda się to na szybsze zużycie takiego układu jego cenę oraz dokładność zliczania. 
Można np. wyobrazić sobie ukłąd bramkujący, który załącza się powyżej pewnego napięcia
$V_G$, jeżeli utrzymuje się ono powyżej tego poziomu przes czas większy niż czas reakcji 
obwodu bramkującego $t_{dG}$. Czas ten jest związany przede wszystkim z indukcyjnością 
obwodu bramkujacego. Jeśli czas trwania sygnału będzie mniejszy niż  $t_{dG}$, 
impuls nie zostanie wykryty, a więc nie zliczony przez licznik.
Należy pamiętać, że na etapie projektowym montażu przetwornika nie zawsze znana jest energia
wymuszenia mechanicznego. Na pewno jeszcze trudniejsze do okereślenia są straty tejże energii
podczas zderzenia. Powyższe świadczy, że układ przetwornika powininen być odpowiednio
przewymiarowany ( należy przyjąć współczynnik korekcji ), aby uniknąć błędu zliczania.  

\begin{table}[h]
  \caption{Ranking optymalizacji pod względem czasu trwania sygnału $dt$}
  \label{fig:sensor_selection_rank_dt}

  \centering
  \pgfplotstabletypeset[
  col sep = comma,
  columns = {Np,probka,dt},
  sort,
  sort cmp={float <},
  sort key=dt,
  every head row/.style={before row=\toprule,after row=\midrule},
  every last row/.style={after row=\bottomrule},
  display columns/0/.style={column name={Np.}, column type = {r}},
  display columns/1/.style={string type,column name={Próbka}, column type = {l}},
  display columns/2/.style={column name={$dt [\si{\milli\second}]$},
  precision=2,fixed zerofill, column type = {r}, fixed},
  skip rows between index={4}{20}
  ]{selekcja_czujnika_dt.csv}
\end{table}

\indent Przy wyborze przetwornika PVDF o najkorzystniejszych parametrach zdecydowano
się wziąć pod uwagę jeszcze jedną właściwość. Nazwano ją subiektywną oceną jakości 
sygnału, czyli kolokwializując wartość "nieodkształcenia sygnału". Jak wcześniej wspomniano
przyjmuje ona wartośći z przedziału \textbf{$0 \div 5$}, gdzie niższa ocena oznacza
gorzszą jakość sygnału. Dla poszczególnych modeli przetworników, a właściwie próbek sygnału 
sporządzono ranking jakości i dla pierwszych 5 pozycji przedstawiono w tablicy 
\ref{fig:sensor_selection_rank_qa}. Tu przodują czujniki oznaczone \textbf{5}. oraz \textbf{6}., 
ale nie bez znaczenia pozostaje wymieniany coraz częściej numer \textbf{1.1}.
 Pełne statystyki znajdują się w ujęciu graficznym na rys. \ref{fig:sensor_statistic}. 


\begin{table}[h]
  \caption{Ranking optymalizacji pod względem oceny jakości sygnału $Q$}
  \label{fig:sensor_selection_rank_qa}

  \centering
  \pgfplotstabletypeset[
  col sep = comma,
  columns = {Np,probka,jakość},
  sort,
  sort cmp={int >},
  sort key=jakość,
  every head row/.style={before row=\toprule,after row=\midrule},
  every last row/.style={after row=\bottomrule},
  display columns/0/.style={column name={Np.}, column type = {r}},
  display columns/1/.style={string type,column name={Próbka}, column type = {l}},
  display columns/2/.style={column name={$Q [b.j.]$ }, column type = {r}},
  skip rows between index={4}{20}
  ]{selekcja_czujnika_qa.csv}
\end{table}

\section{Odpowiedzi elektryczne przetworników}
\label{sec:signal_description}

\indent Podczas wykonywania pomiarów dokonano interesujących obserwacji, o których warto 
wspomnieć. Otóż zauważono zależności kształtu przebiegu sygnału od poszczególnych 
rozwiązań konstrukcyjnych zastosowanych w modelu przetwornika. Pierwszą zauważona relacja
zachodzi pomiędzy stosowaniem tłumienia drgań ( próbki, które w oznaczeniu zawierają małą 
literę "s" ) a czasem gaśnięcia sygnału $t_d$. Po zastosowaniu tłumika drgań w postaci gąbki $t_d$ 
znacząco się skraca. Jest to wniosek dość oczywisty, jednak 
 warto o tym pamiętać konstruując przetworniki piezoelektryczne. Oprócz tłómików czy 
 filtrów elektrycznych są do dyspozycji również mechaniczne. Dla poparcia przedstawionych
 wniosków należy sięgnąć do danych pomiarowych zawartych w talbicy \ref{fig:results_sensor_selection}
 Tak też w układzie z tłumikiem (patrz: Rys.\ref{fig:scope_with_silencer}) czas wygaszania sgnału 
 wynosi $d_t = 14.40 ms$ i jest krótszy niż w układzie bez tłumika 
 $d_t = 164.00 ms$ (patrz: Rys.\ref{fig:scope_without_silencer}.). 
 Znaczna dysproporcja wystąpiła dla każdego kształtu układu pomiarowego.

\indent Analizując dalej wspomniane próbki pomiarów można również dostrzec, że częstotliwość 
podstawowej harmonicznej sygnału ma bardzo zbliżoną wartość. Nie jest to przypadek. 
Dzieje się tak ponieważ jest to główna harmoniczna drgań własnych konstrukcji przetwornika. 
W tym konkretnym przypadku belki o długości $l = 90 mm$ i jednym punkcie mocowania. 
Projektując przetwornik naleźy zwrócić uwagę, aby odstroić drgania własne konstrukcji 
od drgań wymuszających. W innym przypadku uzyskamy szum informacyjny, dojdzie do tak zwanego
zjawiska aliasingu, czyli nałożenia się następujących kolejno po sobie sygnałów. Idąc dalej 
ma to również wpływ na projektowanie elektronicznego układu przetwornika. W obwodach sygnału 
analogowego spodziewa się właśnie częstotliwości drgań własnych przetwornika. W następnym 
rozdziale \ref{sec:construction_optymization} porusza się również zagadnienie zmiany częstotliwości, 
gdy konstrukcja jest belką o dwóch punktach podparcia. 

\indent Napięcie $V_{pp}$ wykazuje się niezmiennością w odniesieniu do przedstawionych do tej pory zmian 
w konstrukcji. Jednak jest ono zależne od zupełnie innego parametru, od energii ciała M 
tuż przed zderzeniem. Jest ona bezpośrednią przyczyną ugięcia belki z przetwornikiem. 
Energia zderzenia, a dokładniej ta stracona, oraz czas kontaktu z belką stanowią ważne parametry 
dla przebiegu sygnału napięciowego przetwornika. Analizując odpowiedź napięciową jednego z przetworników 
(patrz: rys. \ref{fig:scope_with_silencer})
można zauważyć, że w początkowej fazie pierwszy pik napięciowy ma bezwzględną większą wartość niż następny.
Właściwie oba piki można przybliżyć do gasnącego przebiegu sinusoidalnego. Gdy przeanalizowano kolejne okresy
sygnału, spostrzeżono, że każdakolejna półfala ma coraz mniejszą amplitudę. Jest tak, ponieważ przy kolejnych
fazach sygnału belka traci zgromadzoną energię sprężystości. Między innymi jest ona przekształcana w energię
elektryczną generowaną przez piezoelektryk ( więcej na temeat generatorów piezoelektrycznych można znaleźć w 
\cite{To_Do}). Istotnie. Również w przedziale czasowym $0 \div T$ ($T$ - okres
pierwszej harmonicznej sygnału) energia sprężystości belki jest tracona na generację sygnału elektrycznego.
Co dokładnie dzieje się z przepływem energii? Kiedy rysuje się pierwsza półfala sygnału dochodzi do ugięcia
belki przez ciało M. W szczycie napięcia $u_{p}$, który odpowiada maksymalnemu ugięciu belki następuje
zatrzymanie ciała M i cała energia kinetyczna zgromadzona jest w enegii potencjalnej sprężystości belki.
Następnie belka wraca do pozycji neutralnej i energia potencjalna zamienia się w energię kinetyczną ciała M
oraz belki przetwornika. Dalej belka wychyla się w stronę przeciwną, ale już bez kontaktu z ciałem M. Rysuje się 
druga półfala o niższej amplitudzie i przeciwnym znaku. Gdyby pominąć fakt zamiany energii na energię elektryczną
czynną (obwód pomiarowy nieobciążony) oraz brać pod uwagę sygnał dla konstrukcji bez mechanicznego tłumienia
można z przebiegu sygnału obliczyć energię przekazaną do belkowego przetwornika. Otóż spełniona jest
proporcjonalność ze wzoru \ref{eq:energy_prop}. Jeśli wziąć szczyt pierwszej i drugiej półfali
z rys. \ref{fig:scope_with_silencer}, podstawić do \ref{eq:energy_prop} i podzielić oba równania
stronami uzyskamy stosunek energii potencjalnych w pierwszym i drugim półokresie. Otrzymana
w ten sposób wartość stanowi ułamek energii kinetycznej ciała M przed zderzeniem 
(patrz: rys. \ref{fig:mech_char}) przekazanej skonstruowanemu przetwornikowi piezoelektrycznemu.

  
\begin{equation}
E_{ps} /~ u_{p}^2
\label{eq:energy_prop}
\end{equation}
,gdzie $E_{ps}$ - energia potencjalna sprężystości, $u_{p}$ - napięcie szczytowe.

\indent Kiedy otrzymany stosunek oznaczymy przez $k$ to biorc pod uwagę dane zawarte na
rys. \ref{fig:mech_char} ciało M powinno posiadać po zderzeniu energię wyliczoną według
zależności \ref{eq:energy_after_crash}.


\begin{equation}
E_{kE} = (1-k)\dot E_C
\label{eq:energy_after_crash}
\end{equation}

\indent Do powyższych wniosków wraca się również w rozdziale \ref{sec:construction_optymization}.

\begin{figure}[tbph]
\centering
\includegraphics[width=\linewidth]{pictures/1Acs15dt.jpg}
\caption{Przykładowy przebieg pojedynczego sygnału dla próbki oznaczonej 1.1Acs15dt.}
\label{fig:scope_without_silencer}
\end{figure}

\begin{figure}[tbph]
\centering
\includegraphics[width=\linewidth]{pictures/1Acf15dt.jpg}
\caption{Przykładowy przebieg pojedynczego sygnału dla próbki oznaczonej 1.1Acf15dt.}
\label{fig:scope_with_silencer}
\end{figure}
