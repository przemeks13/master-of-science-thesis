\chapter{Metodyka badań}
\label{sec:exp_methods}
Jednym z najważniejszych aspektów badań było zaprojektowanie ich przebiegu. 
 Na Rys.\ref{fig:workflow}. 
 przedstawiono efekt prac nad sposobem realizacji celu eksperymentów. 
 Ważniejszym etapom poświęcono osobny rozdział. Natomiast niewymagającym dłuższego
 komentarza przypisano krótkie wyjaśnienie. 

\begin{figure}[htbp]
\centering
\smartdiagramset{
  set color list={orange, blue,yellow,pink,lime},
  back arrow disabled=true,
  uniform connection color=true,
  description text width=10cm,
  description font = \normalsize,
  module minimum height=1cm,
  descriptive items y sep=2,
  description title width=1cm,
  description title text width=2cm,
  description title font=\scriptsize,
}
% \smartdiagramset{}
\smartdiagram[descriptive diagram]{
{Założenia, Ustalenie założeń konstrukcyjnych czujnika 
(patrz: rozdział \ref{sec:assumptions})},
{Analiza, Separacja założeń mających wpływ na badany układ. },
{Stanowisko, Budowa układu pozwalającego na odizolowanie czynników 
zewnętrznych oraz zapewniającego odpowiednią 
regulację energii zderzenia.(patrz: rozdział \ref{sec:test_stand})},
{Czujniki, Wybór zestawu i rodzaju przetworników do badań.},
{Selekcja czujnika, Badania prowadzące do wyboru optymalnego czujnika.
(patrz: rozdział \ref{sec:sensor_selection})},
{Optymalizacja układu, Eksperymenty prowadzące do optymalizacji przestrzennego 
wyglądu czujnika. Kontrola czujników odrzuconych w kroku poprzednim 
i ewentualny powrót do kroku poprzedniego. 
(patrz: rozdział \ref{sec:construction_optymization})},
{Analiza, Wyznaczenie charakterystyk konstrukcji.
(patrz: rozdział \ref{sec:sensor_characteristic})},
{Analiza, Analiza pozyskanych danych.(patrz: rozdział \ref{sec:conclusion})},
}
\caption{Infografika obrazująca metodykę prowadzonych badań.}
\label{fig:workflow}
\end{figure}

\indent Kluczową rolę w projektowaniu samego przebiegu badań miało przede wszystkim
doprecyzowanie założeń konstrukcyjnych. Pod uwagę wzięto zarówno wejście jak i wyjście
badanego układu. Bardziej doprecyzowane wydaje się być wejście, ponieważ podaje się 
konkretne liczby. Mechanika płynów jest jednak nieprzewidywalnym w $100\%$ zagadnieniem
i dobrze jest przyjąć spory poziom nieufności podczas projektowania. Część dotyczącą
sygnału elektrycznego przedstawiono natomiast ogólnikowo, gdyż układ interpretujący
dopiero zostanie zaprojektowany. Założenia wymagają jedynie zmieszczenia się
parametrów w określonych przedziałach, tak aby zaprojektowanie układu interpretującego
było wykonalne.
\indent Planując metodykę badań skupiono się przede wszystkim na odizolowaniu
środowiska zewnętrznego. Izolacja to nic innego, jak wyodrębnienie sygnału pochodzącego
wyłącznie od przetwornika pobudzonego zderzeniem opisanym w założeniach. Należy tu 
zauważyć, że zbudowane stanowisko pozwoliło osiągnąć szumy na poziomie nie wyższym niż
$40 mV$ i jest to zasługa przede wszystkim wcześniejszego opracowania planu eksperymentów.
