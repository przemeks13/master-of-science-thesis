\chapter{Metodyka badań}
\label{sec:exp_methods}
Jednym z najważniejszych aspektów badań było ich zaprojektowanie. Tu najlepiej sprawdza się metoda burzy mózgów. Można rzec, że była ona nieodłącznym elementem każdego etapu badań. Na Rys.\ref{fig:workflow}. przedstawiono efekt prac nad sposobem realizacji celu eksperymentów. Ważniejszym etapom poświęcono osobny akapit artykułu. Natomiast niewymagającym komentarza przypisano krótkie wyjaśnienie.

\begin{figure}[htbp]
\centering
\smartdiagramset{back arrow disabled=true}
\smartdiagram[descriptive diagram]{
{Założenia, Ustalenie założeń konstrukcyjnych czujnika},
{Analiza, Separacja założeń mających wpływ na badany układ. },
{Stanowisko, Budowa układu pozwalającego na odizolowanie czynników 
zewnętrznych oraz zapewniającego odpowiednią 
regulację energii zderzenia.(patrz: \ref{sec:test_stand})},
{Czujniki, Wybór zestawu i rodzaju przetworników do badań.},
{Selekcja czujnika, Badania prowadzące do wyboru optymalnego czujnika.(patrz: \ref{sec:sensor_selection})},
{Optymalizacja układu, Eksperymenty prowadzące do optymalizacji przestrzennego 
wyglądu czujnika. Kontrola czujników odrzuconych w kroku poprzednim 
i ewentualny powrót do kroku poprzedniego. (patrz: \ref{sec:construction_optymization})},
{Analiza, Analiza pozyskanych danych.(patrz: \ref{sec:conclusion})},
}
\caption{Infografika obrazująca metodykę prowadzonych badań.}
\label{fig:workflow}
\end{figure}
