\chapter{Optymalizacja układu geometrycznego}
\label{sec:construction_optymization}

\newcommand{\tabhead}[1]{\textbf{#1}}

\begin{figure}[htbp]
\centering
\includegraphics[width=\linewidth]{pictures/uklady_geom.pdf}
\caption{Szkic badanych układów konstrukcyjnych sensorów piezoelektrycznych.}
\label{fig:construct_scetch}
\end{figure}
Niełatwe rozważania na temat kierunku badań dotyczących konstrukcji sensora 
zaowocowały ustaleniami o konieczności zbadania przetwornika:
\begin{enumerate}
\item belkowego ze względu na możliwość odniesienia się do istniejącej literatury(patrz: A na Rys.\ref{fig:construct_scetch}),
\item o kształcie rury z powodu przeznaczenia konstruowanego przetwornika (patrz: B na Rys.\ref{fig:construct_scetch}),
\item umieszczonego pomiędzy warstwami miękkiego materiału (patrz: S na Rys.\ref{fig:construct_scetch}),
\end{enumerate}

% \begin{figure}[htbp]
% \centering
\begin{table}[htpb]
    \caption{Wyniki optymalizacji pod względem napięcia międzyszczytowego $V_{pp}$}
    \label{fig:results_opty_vpp}

    \centering
    \pgfplotstabletypeset[
    col sep = comma,
    columns = {probka, vpp},
    sort,
    sort cmp={float >},
    sort key=vpp,
    % string replace*={_}{\textsubscript},
    every head row/.style={before row=\toprule,after row=\midrule},
    every last row/.style={after row=\bottomrule},
    display columns/1/.style={string type,column name={}},
    % create on use/sort order/.style={
    % create col/set list={100,1,2,100,0},
    % },
    create on use/newcol/.style={
        create col/set list={1,2,3}
    },
    columns/newcol/.style={string type},
    columns={newcol,probka,vpp},
    skip rows between index={3}{20}
    ]
    {selekcja_czujnika.csv}
\end{table}

% \end{figure}