\chapter{Optymalizacja układu geometrycznego}
\label{sec:construction_optymization}

\newcommand{\tabhead}[1]{\textbf{#1}}

\begin{figure}[htbp]
\centering
\includegraphics[width=\linewidth]{pictures/uklady_geom.pdf}
\caption{Szkic badanych układów konstrukcyjnych sensorów piezoelektrycznych.}
\label{fig:construct_scetch}
\end{figure}
Niełatwe rozważania na temat kierunku badań dotyczących konstrukcji sensora 
zaowocowały ustaleniami o konieczności zbadania przetwornika:
\begin{enumerate}
\item belkowego ze względu na możliwość odniesienia się do istniejącej literatury(patrz: A na Rys.\ref{fig:construct_scetch}),
\item o kształcie rury z powodu przeznaczenia konstruowanego przetwornika (patrz: B na Rys.\ref{fig:construct_scetch}),
\item umieszczonego pomiędzy warstwami miękkiego materiału (patrz: S na Rys.\ref{fig:construct_scetch}),
\end{enumerate}


\begin{table}[htpb]
    \caption{Ranking optymalizacji konstrukcji pod względem napięcia 
    międzyszczytowego $V_{pp}$ dla czujnika 1.1}
    \label{fig:construction_11_rank_vpp}

    \centering
    \pgfplotstabletypeset[
    col sep = comma,
    columns = {Np, probka, vpp},
    sort,
    sort cmp={int <},
    sort key=Np,
    every head row/.style={before row=\toprule,after row=\midrule},
    every last row/.style={after row=\bottomrule},
    display columns/0/.style={column name={Np.}, column type = {r}},
    display columns/1/.style={string type,column name={Próbka}, column type = {l}},
    display columns/2/.style={column name={$V_{pp} [\si{\volt}]$}, column type = {r}, fixed},
    skip rows between index={5}{25}
    ]
    {geometry_optymalization_11_vpp.csv}
\end{table}

\begin{table}[htbp]
    \caption{Ranking optymalizacji konstrukcji pod względem czasu trwania 
    sygnału $dt$ dla czujnika 1.1.}
    \label{fig:construction_11_rank_dt}
  
    \centering
    \pgfplotstabletypeset[
    col sep = comma,
    columns = {Np,probka,dt},
    sort,
    sort cmp={int <},
    sort key=Np,
    every head row/.style={before row=\toprule,after row=\midrule},
    every last row/.style={after row=\bottomrule},
    display columns/0/.style={column name={Np.}, column type = {r}},
    display columns/1/.style={string type,column name={Próbka}, column type = {l}},
    display columns/2/.style={column name={$dt [\si{\milli\second}]$}}, column type = {r},
    skip rows between index={5}{25}
    ]{geometry_optymalization_11_dt.csv}
\end{table}
  
\begin{table}[htbp]
    \caption{Ranking optymalizacji konstrukcji pod względem 
    oceny jakości sygnału $Q$ dla czujnika 1.1.}
    \label{fig:construction_11_rank__qa}
  
    \centering
    \pgfplotstabletypeset[
    col sep = comma,
    columns = {Np,probka,jakość},
    sort,
    sort cmp={int <},
    sort key=Np,
    every head row/.style={before row=\toprule,after row=\midrule},
    every last row/.style={after row=\bottomrule},
    display columns/0/.style={column name={Np.}, column type = {r}},
    display columns/1/.style={string type,column name={Próbka}, column type = {l}},
    display columns/2/.style={column name={$Q [b.j.]$ }, column type = {r}},
    skip rows between index={5}{25}
    ]{geometry_optymalization_11_qa.csv}
\end{table}

\begin{table}[htpb]
    \caption{Ranking optymalizacji konstrukcji pod względem 
    napięcia międzyszczytowego $V_{pp}$ dla czujnika 6.}
    \label{fig:construction_6_rank_vpp}

    \centering
    \pgfplotstabletypeset[
    col sep = comma,
    columns = {Np, probka, vpp},
    sort,
    sort cmp={int <},
    sort key=Np,
    every head row/.style={before row=\toprule,after row=\midrule},
    every last row/.style={after row=\bottomrule},
    display columns/0/.style={column name={Np.}, column type = {r}},
    display columns/1/.style={string type,column name={Próbka}, column type = {l}},
    display columns/2/.style={column name={$V_{pp} [\si{\volt}]$}, column type = {r}, fixed},
    % skip rows between index={5}{20}
    ]
    {geometry_optymalization_6_vpp.csv}
\end{table}

\begin{table}[htbp]
    \caption{Ranking optymalizacji pod względem 
    czasu trwania sygnału $dt$  dla czujnika 6.}
    \label{fig:construction_6_rank_dt}
  
    \centering
    \pgfplotstabletypeset[
    col sep = comma,
    columns = {Np,probka,dt},
    sort,
    sort cmp={int <},
    sort key=Np,
    every head row/.style={before row=\toprule,after row=\midrule},
    every last row/.style={after row=\bottomrule},
    display columns/0/.style={column name={Np.}, column type = {r}},
    display columns/1/.style={string type,column name={Próbka}, column type = {l}},
    display columns/2/.style={column name={$dt [\si{\milli\second}]$}}, column type = {r},
    % skip rows between index={5}{20}
    ]{geometry_optymalization_6_dt.csv}
\end{table}
  
\begin{table}[htbp]
    \caption{Ranking optymalizacji konstrukcji pod względem 
    oceny jakości sygnału $Q$ dla czujnika 6.}
    \label{fig:construction_6_rank__qa}
  
    \centering
    \pgfplotstabletypeset[
    col sep = comma,
    columns = {Np,probka,jakość},
    sort,
    sort cmp={int <},
    sort key=Np,
    every head row/.style={before row=\toprule,after row=\midrule},
    every last row/.style={after row=\bottomrule},
    display columns/0/.style={column name={Np.}, column type = {r}},
    display columns/1/.style={string type,column name={Próbka}, column type = {l}},
    display columns/2/.style={column name={$Q [b.j.]$ }, column type = {r}},
    % skip rows between index={6}{20}
    ]{geometry_optymalization_6_qa.csv}
\end{table}

