\chapter{Optymalizacja układu geometrycznego}
\label{sec:construction_optymization}

\newcommand{\tabhead}[1]{\textbf{#1}}

\section{Założenia}
\label{sec:construction_asumptions}

Do badań projektowanej konstrukcji wybrano przetwornik \textbf{1.1}. Ze względu na inny sposób montażu
i równie dobre wyniki opisane w podrozdziale \ref{sec:sensor_selection} zdecydowano wykorzystać również
czujnik nr \textbf{6}.

Rozważania na temat kierunku badań dotyczących konstrukcji sensora 
doprowadziły do decyzji o konieczności zbadania przetwornika:
\begin{enumerate}
\item belkowego ze względu na możliwość odniesienia się do istniejącej literatury
\cite{belkowy_sensor}(patrz: A na rys.\ref{fig:construct_scetch}),
\item o kształcie rury z powodu przeznaczenia konstruowanego przetwornika (patrz: B na rys.\ref{fig:construct_scetch}),
\item umieszczonego pomiędzy warstwami miękkiego materiału (patrz: S na rys.\ref{fig:construct_scetch}),
\end{enumerate}
Szczegółowo badane modele przetworników przedstawiono na rys. \ref{fig:construct_scetch}.

\begin{figure}[bthp]
\centering
\includegraphics[width=\linewidth]{pictures/uklady_geom.pdf}
\caption{Szkic badanych układów konstrukcyjnych sensorów piezoelektrycznych.}
\label{fig:construct_scetch}
\end{figure}


\pagebreak
\section{Wybór optymalnego rozwiązania}
\label{construction_selection}

\indent Pierwszym krokiem tego etapu badań było pozayskanie parametrów sygnału dla
różnych układów konstrukcyjnych z zastosowaniem przetwornika \textbf{1.1}. Zebrane
dane w kompletnej formie znajdują się w tablicy \ref{fig:results_11_geometry}. Aby
lepiej zobrazować wyniki pomiarów zaprezentowano je także w formie wykresu słupkowego
( partrz: rys. \ref{fig:construct_11_statistic}). 


\pgfplotstableread[col sep=comma]{geometry_optymization_11.csv}\constructSelTab


\begin{figure}[htbp]
  \centering
  
\pgfplotsset{compat=1.12}
\begin{tikzpicture}

  \begin{groupplot}[
    group style={
        group name=my fancy plots,
        group size=1 by 3,
        xticklabels at=edge bottom,
        xlabels at = edge bottom,
        vertical sep=25pt
    },
    ybar,
    /pgf/bar width=1,
    ymin=0,
    width=\textwidth,
    height=0.4\textwidth,
    axis x line=bottom,
    enlarge x limits=0.25,
    major x tick style = transparent,
    xtick=data,
    xticklabels from table={\constructSelTab}{probka},
    xticklabel style={rotate=60, font=\scriptsize},
    yticklabel style={font=\scriptsize},
    xlabel style={at={(-0.1,-0.05)}, font=\scriptsize},
    xlabel={próbka},
    ylabel style={font=\scriptsize},
    title style={at={(0.5,1.0)}, font=\footnotesize}, 
    grid = major,
    table/x expr=\coordindex,
    table/col sep=comma,
]

\nextgroupplot[ylabel={$V_{pp}[V]$},
               ymax=30,
               ytick distance = 10,
               title={Maksymalne napięcie międzyszczytowe},
              ]
\addplot [fill = red] table[y=vpp]{\constructSelTab};     

\nextgroupplot[ymode=log,
               ylabel={$t_d[ms]$},
               title={Czas trwania sygnału},
               ]
\addplot [fill = green] table[y=dt]{\constructSelTab};;
\nextgroupplot[ymax=5,
               ylabel={$Q[b.j.]$},
               ytick distance = 1,
               title={Ocena jakości sygnału},
               ]
\addplot [fill = blue] table[y=jakość]{\constructSelTab};                 
\end{groupplot}
\end{tikzpicture}

\caption{Zestawienie parametrów sygnałów dla poszczególnych modeli projektowanego przetwornika na
 podstawie tablicy \ref{fig:results_11_geometry}. Czujnik 1.1.}
\label{fig:construct_11_statistic}
\end{figure}

\indent Na wykresie (patrz: rys. \ref{fig:construct_11_statistic}) można zaobserwować,
że rozwiązanie konstrukcyjne A osiąga napięcia międzyszczytowe powyżej $20V$,
co jest najwyższym jest najlepszym rezultatem. Ranking pierwszych pięciu próbek w tym
aspekcie ukazuje tablica \ref{fig:construction_11_rank_vpp}. Konstrukcje o mniejszej 
sztywności pozwalają osiągnąć wyższe wartości $U_{pp}.$ Należy zaznaczyć, że każda
z badanych konstrukcji osiągała napięcie rzędu kilku woltów i więcej.

\begin{table}[h]
    \caption{Ranking optymalizacji konstrukcji pod względem napięcia 
    międzyszczytowego $V_{pp}$ dla czujnika 1.1}
    \label{fig:construction_11_rank_vpp}

    \centering
    \pgfplotstabletypeset[
    col sep = comma,
    columns = {Np, probka, vpp},
    sort,
    sort cmp={int <},
    sort key=Np,
    every head row/.style={before row=\toprule,after row=\midrule},
    every last row/.style={after row=\bottomrule},
    display columns/0/.style={column name={Lp.}, column type = {r}},
    display columns/1/.style={string type,column name={Próbka}, column type = {l}},
    display columns/2/.style={column name={$V_{pp} [\si{\volt}]$}, column type = {r},
    precision=2,fixed zerofill, column type = {r}, fixed},
    skip rows between index={5}{25}
    ]
    {geometry_optymalization_11_vpp.csv}
\end{table}

\indent W tablicy \ref{fig:construction_11_rank_dt} zamieszczono ranking pierwszych
pięciu próbek pod względem czasu tłumienia sygnału. Jak napisano wcześniej, im krótszy
czas tłumienia, tym lepiej. W tym aspekcie najlepiej wypadła konstrukcja B z dwiema
podporami stałymi. Na podstawie danych pomiarowych można stwierdzić, że im mniej
stopni swobody belki z przetwornikiem, tym krótszy czas tłumienia, ale również mniejsza
wartość początkowego napięcia międzyszczytowego.

\begin{table}[h]
    \caption{Ranking optymalizacji konstrukcji pod względem czasu trwania 
    sygnału $dt$ dla czujnika 1.1.}
    \label{fig:construction_11_rank_dt}
  
    \centering
    \pgfplotstabletypeset[
    col sep = comma,
    columns = {Np,probka,dt},
    sort,
    sort cmp={int <},
    sort key=Np,
    every head row/.style={before row=\toprule,after row=\midrule},
    every last row/.style={after row=\bottomrule},
    display columns/0/.style={column name={Lp.}, column type = {r}},
    display columns/1/.style={string type,column name={Próbka}, column type = {l}},
    display columns/2/.style={column name={$dt [\si{\milli\second}]$}, column type = {r},
    precision=2,fixed zerofill, column type = {r}, fixed},
    skip rows between index={5}{25}
    ]{geometry_optymalization_11_dt.csv}
\end{table}

\indent Kolejnym aspektem wyboru optymalnego rozwiązania konstrukcyjnego jest subiektywna
ocena jakości sygnału. Tutaj, większość próbek otrzymało oceny większe lub równe 3. Wyjątkiem
jest model D, który w przypadku zastosowania tłumika z gąbki wygenerował sygnał o jakości $Q = 2$.
Dla formalności zamieszczono w tablicy \ref{fig:construction_11_rank__qa} ranking pierwszych 
pięciu próbek pod względem oceny jakości syganłu.
  
\begin{table}[h]
    \caption{Ranking optymalizacji konstrukcji pod względem 
    oceny jakości sygnału $Q$ dla czujnika 1.1.}
    \label{fig:construction_11_rank__qa}
  
    \centering
    \pgfplotstabletypeset[
    col sep = comma,
    columns = {Np,probka,jakość},
    sort,
    sort cmp={int <},
    sort key=Np,
    every head row/.style={before row=\toprule,after row=\midrule},
    every last row/.style={after row=\bottomrule},
    display columns/0/.style={column name={Lp.}, column type = {r}},
    display columns/1/.style={string type,column name={Próbka}, column type = {l}},
    display columns/2/.style={column name={$Q [b.j.]$ }, column type = {r}},
    skip rows between index={5}{25}
    ]{geometry_optymalization_11_qa.csv}
\end{table}

\indent Podobnie jak dla przetwornika 1.1. na rys. \ref{fig:construct_6_statistic}
przedstawiono uzyskane wartości dla czujnika 6. w postaci diagramu słupkowego.


\pgfplotstableread[col sep=comma]{geometry_optymalization_6.csv}\constructSelTabAny


\begin{figure}[htbp]
  \centering
  
\pgfplotsset{compat=1.12}
\begin{tikzpicture}

  \begin{groupplot}[
    group style={
        group name=my fancy plots,
        group size=1 by 3,
        xticklabels at=edge bottom,
        xlabels at = edge bottom,
        vertical sep=25pt
    },
    ybar,
    /pgf/bar width=1,
    ymin=0,
    width=\textwidth,
    height=0.4\textwidth,
    axis x line=bottom,
    enlarge x limits=0.25,
    major x tick style = transparent,
    xtick=data,
    xticklabels from table={\constructSelTabAny}{probka},
    xticklabel style={rotate=60, font=\scriptsize},
    yticklabel style={font=\scriptsize},
    xlabel style={at={(-0.1,-0.05)}, font=\scriptsize},
    xlabel={próbka},
    ylabel style={font=\scriptsize},
    title style={at={(0.5,1.0)}, font=\footnotesize}, 
    grid = major,
    table/x expr=\coordindex,
    table/col sep=comma,
]

\nextgroupplot[ylabel={$V_{pp}[V]$},
               ymax=30,
               ytick distance = 10,
               title={Maksymalne napięcie międzyszczytowe},
              ]
\addplot [fill = red] table[y=vpp]{\constructSelTabAny};     

\nextgroupplot[ymode=log,
               ylabel={$t_d[ms]$},
               title={Czas trwania sygnału},
               ]
\addplot [fill = green] table[y=dt]{\constructSelTabAny};;
\nextgroupplot[ymax=5,
               ylabel={$Q[b.j.]$},
               ytick distance = 1,
               title={Ocena jakości sygnału},
               ]
\addplot [fill = blue] table[y=jakość]{\constructSelTabAny};                 
\end{groupplot}
\end{tikzpicture}

\caption{Zestawienie parametrów sygnałów dla poszczególnych modeli projektowanego 
przetwornika na podstawie tablicy \ref{fig:results_6_geomrtry}. Czujnik 6.}
\label{fig:construct_6_statistic}
\end{figure}

\indent Kolejno zaprezentowano również rankingi w postaci tabelarycznej. Analia wyników
pokazuje, że najwyższą wartość napięcia międzyszczytowego $U_{pp}$ uzyskano dla modelu E
(patrz: tablica \ref{fig:construction_6_rank_vpp}). Natomiast w przypadku czasu tłumienia
sygnału ponownie najlepszy rezultat uzyskał model B 
(patrz: tablica \ref{fig:construction_6_rank_dt}). Z kolei ocenę jakości sygnału trudno
jest porównać (patrz: tablica \ref{fig:construction_6_rank_qa}) i sprawiedliwie należy
powiedzieć, że jakość każdego z otrzymanych sygnałów jest dobra i porównywalna.  


\begin{table}[h]
    \caption{Ranking optymalizacji konstrukcji pod względem 
    napięcia międzyszczytowego $V_{pp}$ dla czujnika 6.}
    \label{fig:construction_6_rank_vpp}

    \centering
    \pgfplotstabletypeset[
    col sep = comma,
    columns = {Np, probka, vpp},
    sort,
    sort cmp={int <},
    sort key=Np,
    every head row/.style={before row=\toprule,after row=\midrule},
    every last row/.style={after row=\bottomrule},
    display columns/0/.style={column name={Lp.}, column type = {r}},
    display columns/1/.style={string type,column name={Próbka}, column type = {l}},
    display columns/2/.style={column name={$V_{pp} [\si{\volt}]$}, column type = {r},
    precision=2,fixed zerofill, column type = {r}, fixed},
    % skip rows between index={5}{20}
    ]
    {geometry_optymalization_6_vpp.csv}
\end{table}

\begin{table}[h]
    \caption{Ranking optymalizacji pod względem 
    czasu trwania sygnału $dt$  dla czujnika 6.}
    \label{fig:construction_6_rank_dt}
  
    \centering
    \pgfplotstabletypeset[
    col sep = comma,
    columns = {Np,probka,dt},
    sort,
    sort cmp={int <},
    sort key=Np,
    every head row/.style={before row=\toprule,after row=\midrule},
    every last row/.style={after row=\bottomrule},
    display columns/0/.style={column name={Lp.}, column type = {r}},
    display columns/1/.style={string type,column name={Próbka}, column type = {l}},
    display columns/2/.style={column name={$dt [\si{\milli\second}]$}, column type = {r},
    precision=2,fixed zerofill, column type = {r}, fixed},
    % skip rows between index={5}{20}
    ]{geometry_optymalization_6_dt.csv}
\end{table}
  
\begin{table}[h]
    \caption{Ranking optymalizacji konstrukcji pod względem 
    oceny jakości sygnału $Q$ dla czujnika 6.}
    \label{fig:construction_6_rank__qa}
  
    \centering
    \pgfplotstabletypeset[
    col sep = comma,
    columns = {Np,probka,jakość},
    sort,
    sort cmp={int <},
    sort key=Np,
    every head row/.style={before row=\toprule,after row=\midrule},
    every last row/.style={after row=\bottomrule},
    display columns/0/.style={column name={Lp.}, column type = {r}},
    display columns/1/.style={string type,column name={Próbka}, column type = {l}},
    display columns/2/.style={column name={$Q [b.j.]$ }, column type = {r}},
    % skip rows between index={6}{20}
    ]{geometry_optymalization_6_qa.csv}
\end{table}

\indent Na podstawie rozważań prowadzonych w tym podrozdziale zdecydowano, że 
przedstawione zostaną charakterystyki dla 3 układów konstrukcyjnych:
\begin{itemize}
    \item odwrócony B z czujnikiem 1.1 i dwiema podporami stałymi - model rury,
    \item S z czujnikiem 1.1 oraz
    \item odwrócony B z czujnikiem 6 umieszczonym centralnie i dwiema podporami stałymi.
\end{itemize}