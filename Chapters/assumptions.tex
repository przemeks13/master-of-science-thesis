\chapter{Założenia prowadzonych badań}
\label{sec:assumptions}

Jak wspomniano w punkcie \ref{sec:thesis_goal} celem jest konstrukcja przetwornika 
zamieniającego sygnał mechaniczny na elektryczny. Poniżej możliwie dokładnie opisano 
założenia projektowe. Zostały one ustalone w fazie początkowej badań, więc niektóre
dane mogą być nadmiarowe. 

Źródłem wymuszeń mechanicznych są owalne ciała (bryły sztywne) o masie $m_s=0.03\div1.10 g$ 
poruszające się torem ruchu przedstawionym na Rys.\ref{fig:route}. 
Tor wykonany jest ze stalowej rury i w obszarze A (patrz: Rys.\ref{fig:route}) 
następuje sprężysty kontakt ze ścianą toru. Należy nadmienić, że w obszar A uderza 98\% 
poruszających się ciał. W odrębnych badaniach ustalono również, że prędkość ciała 
w momencie kontaktu wynosi $v_s=3.0\div7.0$ $\frac{m}{s}$. Wymuszenia mogą pojawiać 
się minmalnie w odstępach $T_{smin}= 100 ms$.

\begin{figure}[htbp]
\centering
\includegraphics[width=\linewidth]{pictures/tor_lotu_druk.pdf}
\caption{Zakładany tor lotu ciała fizycznego}
\label{fig:route}
\end{figure}

Założono, że miejscem montażu przetwornika jest obszar A na Rys.\ref{fig:route}, który 
stanowi okrąg o średnicy $d_p=70 mm$. Dodatkowo promień ugięcia płaszczyzny A wynosi 
$R_A=140 mm$, a kąt padania ciała na tę powierzchnię $\delta_p=45^{\circ}$. 

\indent Nie bez znaczenia jest rownież wyjście elektryczne układu. Sprecyzowano, że 
otrzymana odpowiedź elektryczna będzie wejściem do czwórnika analogowo cyfrowego o jednym
poziomie kwantyzacji (patrz: oznaczenie AC na rys.\ref{fig:electrical_scheme}). 
Oznacza to, że na wyjściu z tego czwórnika pojawi się logiczna 1, gdy
przetwornik piezoelektryczny będzie w stanie wzbudzenia oraz logiczne 0, gdy przetwornik
piezoelektryczny będzie w stanie spoczynku. Dalej cyfrowy sygnał zostanie przekazany na 
nóżki mikrokontrolera. Na rys.\ref{fig:electrical_scheme} pokazano schemat elektryczny 
przewidywanego układu oraz miejsce (litera T), które stanowi punkt wyjścia dla 
przeprowadzonych badań. Nie sprecyzowano, czy przetwornik AC będzie reagować na zbocze
narastające, opadające, czy może na graniczne wartości napięcia. Wiadomo jednak, że
otrzymany sygnał powinien posiadać wartość kilku woltów. Jest to istotne z punktu widzenia
szumów występujących w rzeczywistym układzie. Przetwornik może przenosić też sygnał drogą
częstotliwościową, jednak nie jest to pożądane i zostało poruszone w roli drugoplanowej.
Dodatkowo założono dla bezpieczeństwa, że sygnał elektryczny powinien gasnąć po 
maksymalnie $50 ms$. Gasnąć, to znaczy zmniejszać się o 3dB w stosunku do minimalnej 
otrzymanej wartości napięcia $V_{pp}$ dla wybranej charakterystyki $V_{pp}(E_k)$ (patrz:
rozdział \ref{sec:sensor_characteristic}). Powodem tego założenia jest reakcji układu
AC, konkretnie elementu bramkującego oraz częstotliwości taktowania mikrokontrolera 
(odbiorcy sygnału cyfrowego).

\begin{figure}[htbp]
\centering
% \fbox{SCHEMAT ELEKTRYCZNY STANOWISKA}
\includegraphics[width=\linewidth]{pictures/blokowy_elektryczny.jpg}
\caption{Schemat blokowy przewidywanego układu.}
\label{fig:electrical_scheme}
\end{figure}

\indent Tak ustalone założenia pozwalają zbadać zależność odpowiedzi 
elektrycznej wybranych przetworników piezoelektrycznych z energią wymuszenia mechanicznego 
a przede wszystkim kostrukcją (zwaną także geometrią) układu.