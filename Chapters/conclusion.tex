\chapter{Podsumowanie}
\label{sec:conclusion}

\indent Przeprowadzone badania miały na celu ocenę możliwości konstrukcji przetwornika 
do określonego zastosowania w przemyśle. Podstawowym wnioskiem z zaprezentowanej dyskusji
wyników jest pozytywna ocena możliwości zastosowania przetworników PVDF we wspomnianym
projekcie. Jak napisano w rozdziale \ref{sec:sensor_characteristic} wyniki zbliżone do
oczekiwanych uzyskano dla czujnika 1.1. (patrz: rys. \ref{fig:sensors}) w układzie 
konstrukcyjnym B (model rury), mocowany do dówch stałych podpór
 oraz S (patrz: rys. \ref{fig:sensor_sel_geometry}). Równie dobre
resultaty dał czujnik 6. w konfiguracji B, także mocowanej do dwóch stałych podpór.
sam czujnik 6., był natomiast mocowany centralnie do konstrukcji B, pozostawiając 
masę rezonansową swobodną. Dla wszystkich trzech wymienionych konstrukcji możliwe jest
wykonanie odpowiedniego przetwornika. Układ oczywiście poza częścią mechaniczną
wymaga czwórnika w postaci filtra oraz układu bramkującego, jak pokazano to na 
rys. \ref{fig:electrical_scheme}. Wyjście z takiego przetwornika byłoby sygnałem 
prostokątnym o zadanej aplitudzie. Realizowane byłoby to poprzez podłączenie zasilania
do konektora $V_{cc}$. Rozwojowo należeałoby prowadzić te badania właśnie w kierunku
konstrukcji odpowiedniego filtra i obwodu bramkującego. 

\indent Omawiana konstrukcja przetwornika mogłaby mieć zastosowanie w przemyśle na
szerszą skalę do detekcji sygnałów mechanicznych wolno zmiennych, a po pewnych 
modyfikacjach nawet szybkozmiennych. Szczególnie trzeba uwzględnienić sygnały w postaci
udarów mechanicznych.
Jedynym wymaganiem jest kontakt przetwornika z elementem maszyny, w którym udar mółgłby
nastąpić. Przetwornik taki może być przytwierdzany również do pudła rezonatorowego
i detekcja wystąpienia udaru mogłaby następować bez bezpośredniego kontaktu. Jednak 
należałoby wtedy odpowiednio zaprojektować ukłąd elektryczny przetwornika, tak aby nasłuch 
był skupiony, nie tyle na amplitudzie sygnału, co częstotliwości. Takie urządzenia, zwane
analizatorami wibroakustycznymi, ocziwiście istnieją, ale są one zazwyczaj mocno rozbudowane.
Celem powyższych badań jest zdecydowanie prostota i optymalizacja ceny produkcji.


